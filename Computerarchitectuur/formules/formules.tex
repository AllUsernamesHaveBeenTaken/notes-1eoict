\documentclass[11pt]{article}
\usepackage{graphicx}
\usepackage[dutch]{babel}
\usepackage{amsthm, amsmath, enumitem, fullpage, helvet, color}
\usepackage[parfill]{parskip}
\let\originalitem\item
\renewcommand{\item}{\originalitem[]}
\renewcommand*{\familydefault}{\sfdefault}
\renewcommand{\d}{\textrm{d}\,}
\usepackage{mathtools}


\usepackage[active,tightpage]{preview}
\renewcommand{\PreviewBorder}{1in}
\newcommand{\Newpage}{\end{preview}\begin{preview}}

\title{Formules Computerarchitectuur}
\author{Haroen Viaene}
\date{19 januari 2015}


\begin{document}
\begin{preview}

\selectlanguage{dutch}

\maketitle

\Newpage

\section{Computervoeding}

\begin{itemize}
	\item $P = U * I$ (vermogen = spanning * stroom)
	\item $Q = I * t$ (capaciteit = stroom * t)
\end{itemize}

\Newpage

\section{Geheugen}

\begin{itemize}
	\item Localiteitsprincipes
	\begin{itemize}
		\item tijdsgebonden: je zal iets niet lang erna opnieuw nodig hebben
		\item plaatsgebonden: je zal de volgende data ook nodig hebben
	\end{itemize}
	\item 
\end{itemize}

\subsection{RAM}

\begin{itemize}
	\item DIMM: twee kanten werken apart, SIMM: onderling verbonden (omkeerbaar)
	\item statisch: flipflop (cache), dynamisch condensator (destructief, herladen) (RAM)
\end{itemize}

\end{preview}
\end{document}