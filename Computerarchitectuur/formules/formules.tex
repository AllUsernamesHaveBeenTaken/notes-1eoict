\documentclass[11pt]{article}
\usepackage{graphicx}
\usepackage[dutch]{babel}
\usepackage{amsthm, amsmath, enumitem, fullpage, helvet, color}
\usepackage[parfill]{parskip}
\let\originalitem\item
\renewcommand{\item}{\originalitem[]}
\renewcommand*{\familydefault}{\sfdefault}
\renewcommand{\d}{\textrm{d}\,}
\usepackage{mathtools}


\usepackage[active,tightpage]{preview}
\renewcommand{\PreviewBorder}{1in}
\newcommand{\Newpage}{\end{preview}\begin{preview}}

\title{Formules Computerarchitectuur}
\author{Haroen Viaene}
\date{19 januari 2015}


\begin{document}
\begin{preview}

\selectlanguage{dutch}

\maketitle

\Newpage

\section{Computervoeding}

\begin{itemize}
	\item $P = U * I$ (vermogen = spanning * stroom)
	\item $Q = I * t$ (capaciteit = stroom * t)
\end{itemize}

\Newpage

\section{Geheugen}

\begin{itemize}
	\item Localiteitsprincipes
	\begin{itemize}
		\item tijdsgebonden: je zal iets niet lang erna opnieuw nodig hebben
		\item plaatsgebonden: je zal de volgende data ook nodig hebben
	\end{itemize}
	\item 
\end{itemize}

\subsection{RAM}

\begin{itemize}
	\item DIMM: twee kanten werken apart, SIMM: onderling verbonden (omkeerbaar)
	\item statisch: flipflop (cache), dynamisch condensator (destructief, herladen) (RAM)
	\item $\frac{\frac{bytes}{transfer}\cdot \frac{transfers}{burst}\frac{cycli}{sec}}{\frac{cycli}{burst}}$ vb: $\frac{8 (=64 bit) \frac{bytes}{transfer}\cdot 4 (=1+1+1+1) \frac{transfers}{burst} 60 MHz \frac{cycli}{sec}}{14 (=5+3+3+3) \frac{cycli}{burst}}$
	\begin{itemize}
		\item FP-DRAM (fast pace): schrijft de paar volgende lijnen ook
		\item EDO-RAM (extended data out): data blijft staan op de uitgang terwijl nieuwe al klaargezet wordt
		\item SD-RAM (Synchronous DRAM): volgende kolommen worden sowieso gelezen op de snelheid van de klok
		\item DDR (Double Data Rate): hogere kloksnelheid, idem voor DDR2,3,4
	\end{itemize}
	\item optimalisatie
	\begin{itemize}
		\item Interleaving: wissel af tussen twee banken om hersteltijd te vermijden
		\item Dual channel: gebruik twee banken tegelijk
		\item Buffered: slaat alle data op op een cache
		\item is van toepassing op alle systemen
	\end{itemize}
\end{itemize}

\Newpage

\subsection{Cache}

\begin{itemize}
	\item Vervangen cache
	\begin{itemize}
		\item FIFO: first in first out 
		\item LRU: least recently used
		\item LFU: least frequently used
		\item ARC: adaptive replacement cache (LFU en LRU combinatie)
	\end{itemize}
	\item types cache
	\begin{itemize}
		\item fully associative: elk deeltje vd RAM heeft een plaats
		\item direct mapped:  elk deel vd geheugenbank heeft 1 plaats in cache, als overschreven wordt is algoritme nodig
		\item set-associative: RAM opgedeeld in sets, binnen die sets wordt overschreven
	\end{itemize}
\end{itemize}

\subsection{formule}
\begin{itemize}
	\item $lines = \frac{cache}{linesize}$
	\item $sets = \frac{linesize}{wayset}$
	\item $sets = 2^X \, en \, linesize = 2^Y$
	\item $taggrootte = (bussize - X - Y) \cdot lines$
	\item toegangstijd: $t_{totaal} = HR_{L1} \cdot t_{L1} + (1 \minus HR_{L1}) \cdot HR_{L2} \cdot t_{L2} + (1 \minus HR_{L1}) \cdot (1 \minus HR_{L2}) \cdot t_{RAM} $
\end{itemize}


\Newpage

\section{Data}

\begin{itemize}
	\item sector: kleinste hoeveelheid door kop leesbaar
	\item track: lijn op schijf
	\item schijf: \dots
	%%%TO FINISH SOON
	\begin{itemize}
		\item FCFS: first come first serve
		\item SSTS: shortest seek time first
		\item SCAN: van 0 $\rightarrow$ einde, doet in terugkeren waaraan er toch gepasseerd wordt
		\item C-SCAN: slaat alles over in terugkeren
		\item LOOK: SCAN, maar gaat niet tot einde
		\item C-LOOK: LOOK, maar slaat alles over in terugkeren
	\end{itemize}
\end{itemize}

\subsection{bestandssystemen}

HALP %%to finish

\end{preview}
\end{document}